\section{\acl{ict:IPC}}
\label{sec:ipc}


``Networking is \ac{ict:IPC} \lcncite{cite:inproceedings:1970:metcalf-1972-acm}''.
``Networking is Distributed \ac{ict:IPC}, and only \ac{ict:IPC} \lcncite{cite:book:2000:day-2007-pna}''.
In short: A \textit{process} is the instance of an application running on a \textit{system}.
A \textit{layer} is a collection of cooperating IPC \textit{processes}, a \ac{org:psoc:rina:DIF}.
A \textit{system} executes \ac{ict:IPC} \textit{processes}.
All \textit{layers} have the same functionality.
They differ in scope and \ac{ict:net:QoS}.

\acs{org:psoc:RINA} is an architecture for \ac{ict:IPC} based on science and a corresponding theory.
The theory is developed in \lcncite{cite:book:2000:day-2007-pna}, often refered to as \ac{people:day-john:PNA}.
The architecture and its specifications are developed, maintained, and published by \ac{org:PSOC}.

There are only two protocols in \acs{org:psoc:RINA}.
The \ac{org:psoc:rina:EFCP} handles flow control.
It has a mandatory part called \ac{org:psoc:rina:DTP} and an optional part called \ac{org:psoc:rina:DTCP}.
\ac{org:psoc:rina:CDAP} is the application protocol, i.e. the only application protocol required for \ac{ict:IPC}.


\subsection{Complexity versus Variance}

The actual problem is variance, not complexity.
Using the principle (or abstraction) of separating `mechism' from `policy' can help to minimize variance.
It does not necessarily minimize complexity.
To avoid `reudctio ad absurdum', we can use a simple guideline:
a `mechanism' focuses on the purpose (of something) and `policies' focus on ways to realize it.
The goal then is to maximize invariance (`mechanisms') and to minimize variance (`policies').
We need to look at commonalities and maximize them, thus reducing \acs{biz:OPEX} and \acs{biz:CAPEX} while improving management.
Minimizing discontinuities also helps to avoid cumbersome solutions.


\subsection{Futher Reading}

There is much more to read, especially in science (yes, there is science).
A good start is `Cybernetics', for instance with work of Ashby, Wiener, Maturana, and von Förster.
Going further back, one can look at philosophy and physics and mathematics.
A good starting point is to look at the work of Newton, Leibnitz, Russel, and Wittgenstein.
Also Frege, Fleck, Kuhn, Popper, Lakatos, Kaufmann, and Dijkstra.
For networking, we would recommend Hansen, Wulf, Watson, Bachman, Metcalf, Shoch, Saltzer, Pouzin, and Day.
For an extended reading list with many links and some discussions see
    \href{https://vdmeer.github.io/skb/research-notes-ina.html}{research-notes-ina}.
