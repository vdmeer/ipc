\section{Production}
\label{sec:production}

    The \ac{people:vdmeer-sven:LCN} main guide and the layouts are created using Microsoft Powerpoint and then exported to \acs{org:iso:PDF}.
    This \ac{people:vdmeer-sven:LCN} Postscript document is created using \LaTeX.
    The website is created using Apache Maven and the Maven-Site plugin.
    All tools and external artifacts are listed in \autoref{sec:material}.


\subsection{Text Files}
    All text files are encoded in the \acs{ict:unicode:UTF}-8 with UNIX line terminators.


\subsection{Graphics}

    The \ac{people:vdmeer-sven:LCN} drawings are all realized using Inkscape.
    The file format is Inkscape SVGZ, essentially \ac{org:w3c:SVG} with some extensions and GZIP compression.
    The font used for characters is Calibri.
    It is not embedded in the source.
    This means editing the source files requires a legal installation of the Calibri font, which comes as standard on Windows.
    \index{SVG}

    Vector images are then exported to \ac{com:us:microsoft:EMF} for use in Microsoft Powerpoint\footnote{An export as vector image to \acs{org:iso:PDF} is also possible for use in other publishing environments, such as \LaTeX.}.
    \ac{org:ietf:PNG}, a bitmap image format defined by the \acs{org:IETF}, is used in the website and for some screen shots.
    \index{EMF}\index{PNG}

    Some graphics show \ac{org:omg:UML} diagrams, for instance to show the elements of an application.
    \ac{org:omg:UML} is a general purpose modeling language used to visualize the design of a software system defined by the \acs{org:OMG}.
    These diagrams are drawn using PlantUML, then exported to \ac{org:w3c:SVG} for some postprocessing.
    \index{UML}


\subsection{Specifications}

    The \ac{people:vdmeer-sven:LCN} provides some specifications, for instance for network schemas and fingerprints.
    Data serialization languages are used to express some of those specifications.
    The first one is \acs{org:ecma:JSON}\footnote{\ac{org:ecma:JSON}, see \href{https://www.json.org/}{json.org}}.
        It was originally defined for JavaScript and is today maintained by the \acs{org:ECMA}.
    Most specifications are presented in \acs{ict:lang:YAML}\footnote{\ac{ict:lang:YAML}, see \href{https://yaml.org/}{yaml.org}}.
        This language, as per version 1.2, is a superset of \ac{org:ecma:JSON}.
    Some of the simple specifications use \ac{ict:CSV}.
        This is a delimited text file that uses a comma to separate values.
    \index{JSON}\index{YAML}\index{CSV}


\subsection{Guide and Layouts}
    Guide and Layouts are realized in Microsoft Powerpoint, using are simple, specially designed template.
    The main font is Bell MT.
    For some parts, Garamond is used.
    None of the fonts are embedded in the \acs{com:us:microsoft:PPTX} sources.
    This means editing the sources requires a legal installation of these fonts, which come as standard on Windows.
    The provided \acs{org:iso:PDF} export has the fonts embedded.

    The template colors are taken from the warm beige colors in \href{https://encycolorpedia.com/cfb997}{encycolorpedia.com/cf997}.
    Advice for creating slides was taken from Jean-luc Doumont, especially this \href{https://www.youtube.com/watch?v=meBXuTIPJQk}{talk}
        and this \href{http://www.principiae.be/X0800.php}{website}.
    Advice for colors and how to use them in graphics (and charts) was taken from
        this \href{https://medium.com/@Elijah_Meeks/color-advice-for-data-visualization-with-d3-js-33b5adc41c90}{blog},
        from the \acs{org:NASA} Earth Observatory blogs on \href{https://earthobservatory.nasa.gov/blogs/elegantfigures/}{elegantfigures} and \href{https://earthobservatory.nasa.gov/blogs/elegantfigures/category/color/}{color}, and
        \href{https://socviz.co/refineplots.html}{section 8} of a book.


\subsection{Postscript}
    The Postscript document is realized using \LaTeX.
    It uses the same fonts as the Guide and Layouts.
    This means that LuaTeX is required, along with a legal installation of these fonts.

    The \LaTeX packages used are standard packages of a TeX Live installation.
    The reference backend is Biber.
    References are processed using Biblatex.
    The index backend is Makeindex.

