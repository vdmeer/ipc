\begin{description}[font=\normalfont,itemsep=0pt]
    \item[\textit{Layer}]
        Communication network layer: \acs{org:psoc:rina:DAF} (application, general), \acs{org:psoc:rina:DIF} (network, specialized DAF).
        A layer is a distributed application

    \item[\acl{ict:net:PHY}]
        Or \acs{ict:net:PHY}, a layer that provides a physical connection, underlying communication system
        \begin{itemize}[noitemsep,topsep=0pt]
            \item e.g. a network technology (Ethernet, WiFi, etc.)
            \item e.g. a (transport) protocol providing communication (\acs{org:ietf:ip:TCP}, \acs{org:ietf:ip:UDP}, GDP, etc.)
        \end{itemize}

    \item[\textit{Process}]
        Instantiation of program executing in a processing system, intended to achieve some purpose.
        Cooperating processes form distributed applications:
            \acs{org:psoc:rina:DAP} (application, general) \acs{org:psoc:rina:IPCP} (network, specialized \acs{org:psoc:rina:DAP}).

    \item[\textit{System}]
        A system that provides resources to execute and manage processes, types are:
        \begin{itemize}[noitemsep,topsep=0pt]
            \item Computing System: has (and manages) one or more processing systems
            \item Processing System: has resources, executes and manages processes
        \end{itemize}
        This document shows only processing systems

    \item[\acl{ict:IF} (\acs{ict:IF})]
        An \acs{ict:API} interface to a blackbox, e.g. a \ac{ict:net:NIC}

    \item[Link, Connection]
        Line between two processes indicating that they are linked; in the PHY layer a physical connection

    \item[Topology]
        Used only in the mathematical sense (homeomorphism), i.e. a network do not have a topology

\end{description}




