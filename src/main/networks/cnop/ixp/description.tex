This is an example of an interconnection or \ac{ict:net:IXP} network.
An \ac{ict:net:IXP} is essentially a third party providing interconnection services to service providers and network operators, who usually do host their own equipment (routers) in their facilities.
Using \ac{ict:IPC}, \acp{ict:net:IXP} are not limited to the exchange of traffic of a single type of inter-network.
Every service provider can exchange traffic belonging to more than one \ac{org:psoc:rina:DIF}, for example
    a general purpose \ac{org:psoc:rina:DIF} like \textit{Internet},
    \acs{ict:net:VPN} \acp{org:psoc:rina:DIF},
    \acs{ict:cloud:DC} interconnect, or
    application-optimized \acp{org:psoc:rina:DIF} tat carry for instance interactive high-definition video traffic.

This network design addresses this situation.
An \textit{IXP Fabric} \ac{org:psoc:rina:DIF} transports traffic belonging to multiple \acp{org:psoc:rina:DIF} (in this example structure \textit{HD Voice} and \textit{Internet}) supported by different \ac{org:ietf:ppvpn:PE} routers.
The configuration is very similar, if not the same, as the DC schema in \autoref{sec:examples:af-d22:dc}.
The difference is in what the fabric connects, \ac{org:ietf:ppvpn:PE} routers or servers (and \acp{ict:VM}).
\textit{IXP Fabric} needs to provide performance and connectivity isolation to the different `utility' \acp{org:psoc:rina:DIF}.
