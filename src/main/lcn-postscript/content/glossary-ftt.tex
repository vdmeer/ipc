\textit{Serialization}.
Several data serialization formats are used or referenced.
    \ac{org:ecma:JSON} was originally defined for JavaScript, but is today used in many other applications as well.
    \index{JSON}
    \ac{ict:lang:YAML} is a simplified language that can be easily translated into \ac{org:ecma:JSON}.
    \index{YAML}

\textit{Image and Graphics}.
Several graphic and image formats are used or referenced.
    \ac{org:w3c:SVG} is an \acs{org:w3c:XML}-based vector image format defined by the \acs{org:W3C} \footnote{For \acs{org:w3c:SVG} specifications see \href{https://www.w3.org/Graphics/SVG/}{w3.org/Graphics/SVG/}}.
        All graphics in the \ac{people:vdmeer-sven:LCN} are drawn in \ac{org:w3c:SVG} using Inkscape \footnote{\href{https://inkscape.org/}{inkscape.org}}.
        \index{SVG}
    \ac{org:ietf:PNG} is a bitmap image format defined by the \acs{org:IETF}.
        It is used mainly for the \ac{people:vdmeer-sven:LCN} website and for screen shots.
    \ac{ict:CSV} is a delimited text file that uses a comma to separate values.
        The \ac{people:vdmeer-sven:LCN} uses it in some examples for encoding network schemas.
        \index{CSV}

\textit{Miscellaneous}.
    The \ac{org:omg:UML} is a general purpose modeling language used to visualize the design of a software system.
        It is specified by the \acs{org:OMG}.
        The \ac{people:vdmeer-sven:LCN} uses it for some modeling, especially for applications.
        \index{UML}
    The \ac{org:w3c:XML} is a markup language used to encode documents in a form that is both human-readable and machine readable.
        It is specified by the \acs{org:W3C}.
        \index{XML}
