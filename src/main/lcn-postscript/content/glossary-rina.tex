\acs{org:psoc:RINA} is an architecture for \ac{ict:IPC} based on science and a corresponding theory.
The theory is developed in \lcncite{cite:book:2000:day-2007-pna}, often refered to as \ac{people:day-john:PNA}.
The architecture and its specifications are developed, maintained, and published by \ac{org:PSOC}.

\textit{Protocols}.
There are two protocols in \acs{org:psoc:RINA}.
The first protocol is \ac{org:psoc:rina:EFCP}.
It handles flow control.
\ac{org:psoc:rina:DTP} is the mandatory part of \ac{org:psoc:rina:EFCP}.
\ac{org:psoc:rina:DTCP} is the optional part of \ac{org:psoc:rina:EFCP}.
\ac{org:psoc:rina:CDAP} is the application protocol, i.e. the only application protocol required for \ac{ict:IPC}.

\textit{Layer Management}.
The \ac{org:psoc:rina:RIB} \#\#\#

\textit{Management Application}.
The \ac{org:psoc:rina:DMS} is the management application in \acs{org:psoc:RINA}.
There are four different scopes for a \ac{org:psoc:rina:DMS}:
    \begin{itemize}[topsep=0pt, partopsep=0pt, nosep]
        \item The \ac{org:psoc:rina:dms:OS-DMS} manages the operating system;
        \item The \ac{org:psoc:rina:dms:NM-DMS} manages the network;
        \item The \ac{org:psoc:rina:dms:AM-DMS} manages applications; and
        \item The \ac{org:psoc:rina:dms:NSM-DMS} manages all name spaces.
    \end{itemize}

\textit{Projects}.
Several research funding bodies have invested in \acs{org:psoc:RINA} by financing research and development projects.
    \acs{org:eu:fp7:projects:PRISTINE} was an \acs{org:EU} \acs{org:eu:FP7} project \#\#\#.
    \index{PRISTINE}
    \acs{org:eu:h2020:projects:ARCFIRE} was an \acs{org:EU} \acs{org:eu:H2020} project \#\#\#.
    \index{ARCFIRE}

