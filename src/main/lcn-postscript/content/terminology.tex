\subsection{System, Layer, Process}

    These three terms are the main elements of \ac{ict:IPC}.
    They are written in italic to distinguish them from other uses, e.g. \textit{layer} means the term layer while graphic layer just uses the word layer in a different context.
    The terms are probably of equal importance, depending on what you want to see, describe, define, specifiy, or process.

    A \textit{system} provides resources to execute and manage processes.
    A \textit{Computing System} has (and manages) one or more \textit{Processing System}, which have resources and execute and manage \textit{processes}.
    The \ac{people:vdmeer-sven:LCN} shows mostly \textit{Processing Systems}.
    \index{System}\index{Computing System}\index{Processing System}

    A \textit{layer} is a distributed application.
    \acs{org:psoc:RINA} separates application layers from networking layers.
    A \ac{org:psoc:rina:DAF} is an application layer.
    A \ac{org:psoc:rina:DIF} is an network layer, essentially a specialized version of a \ac{org:psoc:rina:DAF}.
    The lowest \textit{layer} connects our applications to some technology.
    This \textit{layer} is called \textit{Shim} or \textit{Shim \acs{org:psoc:rina:DIF}}.
    It uses an \ac{ict:IF} \footnote{an \ac{ict:API} to a blackbox} to either
        a \ac{ict:net:PHY} (e.g. Ethernet or \acs{org:ieee:802:WIFI}) or
        a (transport) protocol providing communication (e.g. \acs{org:ietf:ip:TCP} or \acs{org:ietf:ip:UDP}).
    \index{Layer}\index{DAF}\index{DIF}\index{Shim}

    A \textit{process} is the instantiation of program executed in a \textit{Processing System}, intended to achieve some purpose.
    Cooperating processes form distributed applications.
    \acs{org:psoc:RINA} separates application-related processes from network-related processes.
    We have a \ac{org:psoc:rina:DAP} for general applications (an application process in a \ac{org:psoc:rina:DAF}) and
        a \ac{org:psoc:rina:IPCP} for networks (an application process in a \ac{org:psoc:rina:DIF}).
    \acp{org:psoc:rina:IPCP} are specialized \acp{org:psoc:rina:DAP}.
    A link or connection between \textit{processes} means that they are connected (note: this is a connection between \textit{processes} and not \textit{systems}).
    A link in the \ac{ict:net:PHY} layer is a physical connection between \textit{systems}.
    \index{Process}\index{DAP}\index{IPCP}

