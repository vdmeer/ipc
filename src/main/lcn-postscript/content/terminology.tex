\section{Terminology}
\label{sec:terminology}


\subsection{System, Layer, Process}

    These three terms are the main elements of \ac{ict:IPC}.
    They are written in italic to distinguish them from other uses, e.g. \textit{layer} means the term layer while graphic layer just uses the word layer in a different context.
    The terms are probably of equal importance, depending on what you want to see, describe, define, specifiy, or process.

    A \textit{system} provides resources to execute and manage processes.
    A `Computing System' has (and manages) one or more `Processing System', which have resources and execute and manage \textit{processes}.
    The \ac{people:vdmeer-sven:LCN} shows mostly `Processing Systems'.
    \acs{org:psoc:RINA} shows that we have only three kinds of systems: hosts, border routers, and interior routers.
    \index{System}\index{Computing System}\index{Processing System}

    A \textit{layer} is a distributed application.
    \acs{org:psoc:RINA} separates application layers from networking layers.
    A \ac{org:psoc:rina:DAF} is an application layer.
    A \ac{org:psoc:rina:DIF} is an network layer, essentially a specialized version of a \ac{org:psoc:rina:DAF}.
    The lowest \textit{layer} connects our applications to some technology.
    This \textit{layer} is called \textit{Shim} or \textit{Shim \acs{org:psoc:rina:DIF}}.
    It uses an \ac{ict:IF} \footnote{an \ac{ict:API} to a blackbox} to either
        a \ac{ict:net:PHY} (e.g. Ethernet or \acs{org:ieee:802:WIFI}) or
        a (transport) protocol providing communication (e.g. \acs{org:ietf:ip:TCP} or \acs{org:ietf:ip:UDP}).
    \index{Layer}\index{DAF}\index{DIF}\index{Shim}

    A \textit{process} is the instantiation of program executed in a \textit{Processing System}, intended to achieve some purpose.
    Cooperating processes form distributed applications.
    \acs{org:psoc:RINA} separates application-related processes from network-related processes.
    We have a \ac{org:psoc:rina:DAP} for general applications (an application process in a \ac{org:psoc:rina:DAF}) and
        a \ac{org:psoc:rina:IPCP} for networks (an application process in a \ac{org:psoc:rina:DIF}).
    \acp{org:psoc:rina:IPCP} are specialized \acp{org:psoc:rina:DAP}.
    A link or connection between \textit{processes} means that they are connected (note: this is a connection between \textit{processes} and not \textit{systems}).
    A link in the \ac{ict:net:PHY} layer is a physical connection between \textit{systems}.
    \index{Process}\index{DAP}\index{IPCP}



\subsection{Graphics}

    The \ac{people:vdmeer-sven:LCN} drawings are all realized in the graphic format \ac{org:w3c:SVG}\footnote{an \acs{org:w3c:XML}-based vector image format defined by the \acs{org:W3C}, for \acs{org:w3c:SVG} specifications see \href{https://www.w3.org/Graphics/SVG/}{w3.org/Graphics/SVG/}}
    The \ac{org:w3c:SVG} files use layers, called `graphic layers' to separate them from \textit{layers}.
    The layer names are long to ease the (manual) authoring, drawing, and export process.
    A short form of the names is called \ac{people:vdmeer-sven:lcn:GLC}.
    It can be used as a standardized form for automated processes an to exchange between software processing \ac{people:vdmeer-sven:LCN} drawings and images.
    \index{SVG}\index{GLC}

    The names of the `graphic layers' and the \ac{people:vdmeer-sven:LCN} guide use a few terms: system, cell, acr, name, frame, edge, and 3D.
    Here, `system' means a processing system (in the graphic a blue, colored, or grey box with a string frame).
    The term `cell' is used for a cell in the rectangular grid (in the guide) and for a graphic layer that marks cells in systems (using horizontal and vertical lines).
    The terms 'acr' and 'name' are used for graphic layers that have acronyms (usually a \acs{org:psoc:RINA}, e.g. \ac{org:psoc:rina:DIF} or \ac{org:psoc:rina:IPCP}) or some name (e.g. the name of a \textit{layer} or a \textit{process}).
    'Frame' is used for objects that do not have a fill.
    An 'edge' is a link between two or more graphic objects, e.g. links between \textit{processes}.
    '3d' is used for \ac{ict:graphic:3D} graphics, mostly isometric cubes, prisms, and cylinders.

    The term topology requires special attention.
    In the \ac{people:vdmeer-sven:LCN} guide, 'topology' is used only in the mathematical sense (homeomorphism).
    So topology refers to graphs with a common property, e.g. star, bus, ring, (fully) mesh(ed), tree, daisy chain, hybrid.
    This is in contrast to the common use in networking where it means the graph of links between network equipment, i.e. not a homeomorphism.



\subsection{Concepts}
    The \ac{people:vdmeer-sven:LCN} uses and intoduces a few concepts with associated terms.
    'Choice' and 'option' refer to graphic mechanisms and graphic policies (as in the principle of separation of mechanism and policy).
    A 'choice' is a graphic mechanism. A set of choices is a set of grpahic mechanisms from which one or more can be selected.
    For instance, some of the graphic layers are choices: \textit{systems} and \textit{layers} and \textit{processes} can be shown separately or in combination.
    An 'option' is a graphic policy used to change the behavior of a choice (if it has options).
    For instance, \textit{systems} can be shown with blue, colored, or grey fill. Same for \textit{layers} and \textit{processes}.
    Some options can be used at the same time, some have to be used exclusively (e.g. fill).

    A '(network) schema' is a schematic representation of a network, focusing on \textit{system} classes and types,
    not showing all instances of \textit{systems} in a network (e.g. only showing the class host but not many hosts).
    A 'network' is a complete network with all systems (e.g. with all \textit{systems}).

    A 'fingerprint' is an abstracted (decreased details) representation of a network or a schema.
    It can be an image or text.

    'Tree' is used in the mathematical sense, a connected graph without cycles.
    When the direction in this graph are important, a \ac{ict:concept:DAG} is used: a finite directed graph with no indirect cycles (tree with directions).


